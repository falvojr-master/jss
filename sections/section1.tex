\section{Introduction}

Constantly changes in software reuse approaches lead to the concept of Software Product Line (SPL), which represents a shift in focus from the singular software development paradigm. From this concept, companies that previously developed project-by-project software have now focused on the creation and maintenance of SPL and their variabilities. Thus, models that represent variabilities are specified as part of the core assets of an SPL, in which their correct identification, specification and representation allow to take several development benefits \cite{chen11, capilla13}.

To manage the variabilities that allow to diversify the portfolio of products in a given domain, it is necessary the adoption of a well-defined systematic approach. Some of these approaches may be used in Domain Engineering (DE) and Application Engineering (AE), supporting the selection and delimitation of the variant artifacts from different products \cite{bockle05, vanderlinden07}.

For an SPL to be successful, its domain must be defined carefully. If the domain is too large and product members vary too widely, the core assets will be strained beyond their ability to accommodate the variation, economies of production will be lost, and the product line will collapse into an old-style, one-at-a-time product development effort. If the domain is too small, the core assets might not be built in a generic enough fashion to accommodate future growth, and the product line will stagnate: economies of domain will never be realized, and the full potential return on investment (ROI) will never materialize \cite{bockle05, vanderlinden07}.

In a different but related perspective, the rapid growth of information and communications technologies has favored the emergence of innovative ways to face the shortcomings of traditional education \cite{west12}. In this context, mobile learning (m-learning) has emerged, providing a strong interaction between learners and instructors, enabling them to actively participate of the knowledge construction process anytime and anywhere \cite{kukulska05}. 

M-learning has grown in terms of importance and visibility, mainly because of significant results with regard to flexibility and propagation of education \cite{kinshuk03, wexler08}. These aspects have made of m-learning a promising tool on behalf of education. In 2011, the first ``UNESCO Mobile Learning Week'' suggested m-learning as an alternative to that which UNESCO called ``teacher crisis'', expression justified by global need for 8.2 million new teachers to achieve the UN Millennium Development Goal of providing universal primary education by 2015 \cite{west12}.

Due to the diversity of platforms, technologies and pedagogical methods that can be considered for the development of m-learning applications, a wide range of specificities  may be streamlined and addressed through a reuse perspective. However, there are few works addressing development issues through a strategy of systematic reuse, such as SPL, for mobile middleware and, consequently, for the m-learning domain as well \cite{bezerra09}.

Motivated by this scenario, we have worked on the establishment of M-SPLearning, an SPL to the m-learning domain~\cite{falvojr14a, falvojr14b}. M-SPLearning has been developed based on a concise UML-based variability management approach, named SMarty~\cite{oliveirajr10}, which provides mechanisms to facilitate the identification and representation of variabilities.

This paper discusses how variability improves the development of m-learning applications by means of the use of SMarty. We experimentally evaluated M-SPLearning with respect to singular software development, particularly comparing time-to-market and quality of the software products implemented from the use of both (SPL and singular) approaches. The results showed a significant reduction of time-to-market and a quality improvement, in terms of faults, when considering the software products developed from the M-SPLearning core assets with the support of variabilities.

The paper is organised as follows.  Background is summarized in Section \ref{section2}, M-SPLearning is described in Section \ref{section3}. In Section \ref{section4}, we present the experimental evaluation performed. In Section \ref{section5} we discuss the lessons learned from the development and application of M-SPLearning. Related work is presented in Section \ref{section6}. Conclusions and perspectives for future work are discussed in Section \ref{section7}.